\begin{abstract}

Printers are a prevalent piece of hardware in all types of networks - home, business, or educational.
They are an invaluable tool in bridging the divide between digital and physical documents, but often times these devices are not given the attention they need.
Thousands of printers can be discovered online through a quick and simple Google search, revealing that these devices do not require any form of authentication and are publicly available.
This leaves them wide open to a variety of exploitations.

In this paper, we show how public-facing printers can be taken advantage of through the use of bitcoin operations.
By using the PostScript language, printers can become slaves to whatever computation is hidden inside of a PostScript file.
To show this, we crafted a PostScript file that computes the SHA-256 function, which is the cornerstone of the bitcoin mining algorithm.

In addition, we show how one can retrieve computational results from a printer without having a 2-way channel of communication available.
Our method of retrieving results involves the use of heuristics to reconstruct the bitcoin values a certain printer has found.
By showing that a printer is capable of these operations, we demonstrate that one can gain considerable computational power by creating a network of printers to do computational operations at the sender’s discretion.

\end{abstract}
