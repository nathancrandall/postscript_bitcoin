\section{Background}

The cryptographic currency, Bitcoin, is based upon the cryptographically secure hashing algorithm, SHA-256, which is part of the SHA-2 family.
SHA-256 takes a variable sized input and returns a unique 256-bit output.
Bitcoin is a decentralized electronic currency that is operated through a peer-to-peer network and aims to prevent double spending through a hash-based proof-of-work.
Bitcoin values are calculated through the following steps: the Bitcoin protocol first finds other Bitcoin nodes to figure out important information such as what Bitcoin block is currently being mined.
After, the block header is constructed.
This header contains 80 bytes of information about the current block, such as the version, timestamp, nonce, and target value (see Table~\ref{bitcoin_header}).

Once all of this information has been assembled, the 80-byte block header is hashed via SHA-256.
The output of this hash is then hashed once more through SHA-256.
If the hash value found is less than the target value, then that signals that a bitcoin value has been found.
Otherwise, the nonce value is incremented and the process is repeated until a bitcoin value is discovered.

Bitcoin self-regulates itself through the use of an ever increasing difficulty.
Every time a block (a block is all of the bitcoin values that can be mined from the same block header)  is mined, the difficulty of finding a proper bitcoin value is increased.
This prevents Moore’s law from ruining the value of Bitcoins; even if hardware power increases exponentially, the difficulty of finding Bitcoins will become exponentially more difficult as well.
Bitcoin values also provide proof-of-work which verifies that the values are legitimate.

\begin{table*}
    \centering
\label{bitcoin_header}
\begin{tabular}{r| p{9cm} | r}
  \hline
  \hline
  {\bf Field} & {\bf Description} & {\bf No. of Bytes} \\
  \hline
  Version & Version of Software & 4 \\
  HashPrevBlock & Hash of the previous block header & 32 \\
  HashMerkleRoot & Hash based on all transactions in the block & 32 \\
  Time & Epoch Time & 4 \\
  Bits & Current target value & 4 \\
  Nonce & 32-bit integer & 4 \\
  \hline
\end{tabular}
\caption{The block header for a Bitcoin. These blocks are hashed twice using SHA-256 to determine whether or not the header (with a given nonce) represents a valid Bitcoin. Because the required value becomes harder and harder to match, more and more computation is required to mine additional Bitcoins.}
\end{table*}

Google is an invaluable tool for locating documents, images, webpages, and network devices.
Google has its own syntax for narrowing searches, such as matching a substring in a certain URL or title.
This feature is what makes accessing different printing devices that are publicly available on the internet such an easy task.
Through the use of Google Dorks, one can refine their search to return URLs that only a specific device would contain, such as a HP Laserjet printer.
There are useful operators such as inurl, intitle, and intext that allow an individual to virtually locate any type of public network printer.
One does not have to look too far, however, since many websites have a comprehensive list readily available for you to copy and paste into Google to locate different devices~\cite{exploit_db}.
Additionally, an attacker can easily locate exploitable and unsecured devices through the use of the Shodan search engine~\cite{shodan}.

\begin{table*}
    \centering
\label{printer_dorks}
\begin{tabular}{r| p{9cm} | r}
    \hline
    \hline
    {\bf Printer series} & {\bf Google search query} & {\bf Printer count} \\
    \hline    
    Dell M5200 & intitle:"Dell Laser Printer M5200" port\_0 & 4 \\
    \hline
    Samsung CLX & allintitle:"SyncThru Web Service" & 8 \\
    \hline
    Xerox Docuprint & intext:"MaiLinX Alert (Notify)" "XEROX CORPORATION. Product and service" & 13 \\
    \hline
    Xerox Phaser 3000 series & "display printer status" intitle:"Home" & 81 \\
    \hline
    Samsung Printers & ""This page is for configuring Samsung Network Printer" | printerDetails.htm" & 402 \\
    \hline
    Xerox Phaser 6250 & "Phaser 6250" "Printer Neighborhood" "XEROX CORPORATION. All Rights Reserved." & 472 \\
    \hline
    HP Laserjet and Brother HL & "inurl:":631/printers" -php -demo" & 901 \\
    \hline
    Brother HL & inurl:"printer/main.html" intext:"settings" & 4,310 \\
    \hline
    Lexmark T series & "intext:"UAA (MSB)"  Lexmark -ext:pdf "Ethernet"" & 16,600 \\
    \hline
    HP Laserjet & inurl:"hp/device/this.LCDispatcher?nav=hp.Print" & 94,000 \\
    \hline
    \hline
    {\bf Total} & & 116,191 \\
    \hline
\end{tabular}
\caption{A set of series of printers that can be located by using Google, their accompanying Google Dorks, and the count of printers returned by the search query.}
\end{table*}
