\section{Introduction}

Printers are pieces of hardware that are never given much attention as long as they are working properly and filled with paper.
Unfortunately, this also means that they are not given much thought when it comes to security.
Modern printers now come packaged with data storage, full Internet-Protocol stacks, and embedded operating systems[1].
Clearly these devices are far more powerful than they used to be just a decade ago.
By accessing a printer, an attacker can glean important information about the network it resides on, such as software versions, firmware versions, and printed documents.
In addition, an attacker can easily perform a denial-of-service attack against the printer with a maliciously crafted PostScript file.
This paper takes a different direction by using the printers for their available resources, rather than their exploitative abilities.

Google reveals that there are approximately 124,000 Hewlett-Packard (HP) printers with public IP addresses right now[2].
This number will continuously grow, since just one month ago there were only 105,000.
If the number of publicly available printers keeps increasing, this means that there is a constantly growing amount of computational power.
This computational power also goes unused, as printing jobs are not computationally intensive most of the time, and printers are only used for specific events; they are sparingly used in terms of time.
This means that these devices are free for the majority of the time that they are active.

In this paper, we investigate the usefulness of using printers as computational hardware in a bitcoin mining operation.
We explore how bitcoin values are constructed through the cryptographic hashing function, SHA-256.
Next, we explain how public printing devices can be found quickly and easily through the use of the Google search engine and its features.
Once these printers are found, we explain how these devices can be given any type of computational task and perform it, all through the use of the page description language, PostScript.
Finally, we review a printer�s performance for the SHA-256 hash function and compare it to other devices that are popular in bitcoin mining communities.

